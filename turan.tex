\documentclass{patmorin}
\listfiles
\usepackage[utf8]{inputenc}
\usepackage{amsthm,amsmath,graphicx}
\usepackage{pat}
\usepackage[letterpaper]{hyperref}
\usepackage[dvipsnames]{color}
\definecolor{linkblue}{named}{Blue}
\hypersetup{colorlinks=true, linkcolor=linkblue,  anchorcolor=linkblue,
citecolor=linkblue, filecolor=linkblue, menucolor=linkblue,
urlcolor=linkblue, pdfcreator=Me, pdfproducer=Me} \setlength{\parskip}{1ex}


\DeclareMathOperator{\sign}{sign}
\DeclareMathOperator{\xmax}{xmax}
\DeclareMathOperator{\xmin}{xmin}
\DeclareMathOperator{\ymax}{ymax}
\DeclareMathOperator{\ymin}{ymin}
\DeclareMathOperator{\survivors}{survivors}

\usepackage{array}


% To reduce space in lists
\usepackage{enumitem}  
\setlist{noitemsep}

%\usepackage[skip=0pt]{caption}

\title{\MakeUppercase{More Turán-Type Theorems for Triangles in Convex Point Sets}\thanks{This research is partially funded by NSERC.}}
\author{Authors TBD}


\newcommand{\taco}{\raisebox{-.1ex}{\includegraphics[height=1.6ex]{figs/triangles-edge-1}}}
\newcommand{\mariposa}{\raisebox{-.1ex}{\includegraphics[height=1.6ex]{figs/triangles-edge-2}}}


\newcommand{\bat}{\raisebox{-.1ex}{\includegraphics[height=1.6ex]{figs/triangles-vertex-1}}}
\newcommand{\nested}{\raisebox{-.1ex}{\includegraphics[height=1.6ex]{figs/triangles-vertex-2}}}
\newcommand{\crossing}{\raisebox{-.1ex}{\includegraphics[height=1.6ex]{figs/triangles-vertex-3}}}

\newcommand{\ears}{\raisebox{-.1ex}{\includegraphics[height=1.6ex]{figs/triangles-disjoint-1}}}
\newcommand{\swords}{\raisebox{-.1ex}{\includegraphics[height=1.6ex]{figs/triangles-disjoint-2}}}
\newcommand{\david}{\raisebox{-.1ex}{\includegraphics[height=1.6ex]{figs/triangles-disjoint-3}}}

\DeclareMathOperator{\ex}{ex}
\DeclareMathOperator{\on}{\overline{ex}}



%\usepackage{lineno}
%\linenumbers

\pagenumbering{roman}
\begin{document}
\begin{titlepage}
\maketitle

\begin{abstract}
  We study the following family of problems: Given a set of $n$ points
  in convex position, what is the maximum number triangles one can create
  having these points as vertices while avoiding certain \emph{forbidden
  configurations}.  As forbidden configurations we consider all 8 ways
  in which a pair of triangles in such a point set can interact.
\end{abstract}

\end{titlepage}

\tableofcontents


\section{Introduction}
\pagenumbering{arabic}

Let $t_1$ and $t_2$ be a pair of distinct triangles whose (4--6) vertices
are in convex position.  There are 8 combinatorially distinct ways that
these triangle can interact:  2 ways in which the triangles can share
an edge (\taco\ and \mariposa), 3 ways in which the triangles can share a
single vertex (\bat, \nested, and \crossing), and 3 ways in which
the triangles can have no vertices in common (\ears, \swords,
and \david).

We consider the following class of problems:  Given a set, $X$,
of combinatorial configurations of pairs of triangles, what is the
largest set, $S$, of triangles one can create whose vertices are $n$
points in convex position, and such that no pair of triangles in $S$
forms a configuration in $X$.  We call the size of this set $\ex(n,X)$.
For example, 
\begin{equation}
    \ex(n,\{\taco,\nested,\crossing,\swords,\david\}) = n-2 \enspace .
\end{equation}
This is because the set
$X=\{\taco,\nested,\crossing,\swords,\david\}$ in this case
forbids any form of crossings between the edges of triangles. Thus,
the number maximum number of triangles we can have while avoiding $X$
is the number of triangles in a triangulation of a convex $n$-gon,
i.e., $n-2$. One of the results in this paper is that nearly the
same bound holds even if we allow the $\swords$ and $\david$
configuration. In particular, our \thmref{blech} shows that
$\ex(n,\{\taco,\nested,\crossing\}) \in O(n\log n)$.

Sometimes it is more natural to describe the allowable configurations than the forbidden configurations. For this, we use the notation 
\[
   \on(n,X) = \ex(n,\{\taco,\mariposa,\bat,\nested,\crossing,
                       \ears,\swords,\david\} \setminus X) \enspace .
\]



\begin{table}
\begin{center}
\begin{tabular}{m{.55\textwidth}m{.4\textwidth}}
  \hline
  \textbf{Result} & \textbf{Ref.} \\ \hline\hline
  $\ex(n,\{\mariposa\})\in \Theta(n^3)$ \newline
  $\ex(n,\{\taco\})\in \Theta(n^2)$ & \cite{brass:turan} \\
  \hline
  $\ex(n,\{\bat\})\in \Theta(n^3)$ \newline
  $\ex(n,\{\nested\})\in \Theta(n^2)$ \newline
  $\ex(n,\{\crossing\})\in \Theta(n^2)$ & \cite{brass:turan} \\
  \hline
  $\ex(n,\{\ears\})\in \Theta(n^3)$ \newline
  $\ex(n,\{\swords\})\in \Theta(n^2)$ \newline
  $\ex(n,\{\david\})\in \Theta(n^2)$ & \cite{brass:turan} \\
  \hline
  $\ex(n,\{\bat,\nested\})\in \Theta(n^2)$ \newline 
  $\ex(n,\{\nested,\crossing\})\in \Theta(n^2)$ \newline
  $\ex(n,\{\bat,\crossing\})\in \Theta(n^2)$ & \cite{brass:turan} \\
  \hline
  $\ex(n,\{\ears,\swords,\bat,\nested\}) = n$ \newline
  $\on(n,\{\taco,\mariposa,\david,\crossing\}) = n$
    & \cite{brass.rote.ea:triangles} \\
  \hline
  $\ex(n,\{\bat,\nested,\crossing\}) \in \Theta(n)$ \newline
  $\ex(n,\{\taco,\mariposa\}) \in \Theta(n^2)$ \newline 
  $\ex(n,\{\ears,\swords,\david\}) \in \Theta(n^2)$ & from hypergraphs \\ \hline
%  $\ex(n,X\subset\{\taco,\mariposa,\bat,\nested,\crossing,\ears,\swords,\david\}) \in \Omega(n)$ & here  \\
%  $\ex(n,\{\taco,\nested\}) \in \Omega(n^{3/2})$ & here \\
%  $\ex(n,\{\taco,\nested,\bat\}) \in \Omega(n^{3/2})$ & here \\ \hline
%  $\ex(n,\{\taco,\mariposa,\nested,\bat,\ears\})
%      \in\Omega(n^{3/2})$ \newline
%  $\on(n,\{\crossing,\swords,\david\}) \in \Omega(n^{3/2})$  
%    & here  \\ \hline
%  $\ex(n,\{\taco,\nested,\crossing\}) \in \Omega(n) \cap O(n\log n)$ & \thmref{blech} \\
%  $\ex(n,X\cup\{\mariposa\})\in \Theta(\ex(n,X))$ & \lemref{xcup} \\
%  \hline
%  $\ex(n,\{\ears,\swords,\nested\}) \in \Theta(n)$
%    & extension of \cite{brass.rote.ea:triangles} \\
%  \hline
  
  \end{tabular}
\end{center}
\caption{Known results on $\ex(n,X)$ for different sets $X$.}
\end{table}


\begin{tabular}{llll}\hline
  $\ex(n,\taco,\bat)$ & & & \\
  $\ex(n,\taco,\nested)$ & $\Omega(n^{\log_9 28})$ & $O(n^2)$ & [here] \\
  $\ex(n,\taco,\crossing)$ & & & \\
  $\ex(n,\taco,\ears)$ & & & \\
  $\ex(n,\taco,\swords)$ & $\Omega(n)$ & $O(n\log n)$ & \thmref{taco-swords} \\
  $\ex(n,\taco,\david)$ & & & \\
  $\ex(n,\bat,\nested)$ & \multicolumn{2}{c}{$\Theta(n^2$)} & \cite{brass:turan} \\
  $\ex(n,\bat,\crossing)$ & \multicolumn{2}{c}{$\Theta(n^2$)} & \cite{brass:turan} \\
  $\ex(n,\bat,\ears)$ & & & \\
  $\ex(n,\bat,\swords)$ & \multicolumn{2}{c}{$\Theta(n^2$)} & \thmref{bat-swords} \\
  $\ex(n,\bat,\david)$ & & & \\
  $\ex(n,\nested,\crossing)$ & \multicolumn{2}{c}{$\Theta(n^2$)} & \cite{brass:turan} \\
  $\ex(n,\nested,\ears)$ & & & \\
  $\ex(n,\nested,\swords)$ & $\Omega(n)$ & $O(n\log n)$ & \thmref{nested-swords} \\
  $\ex(n,\nested,\david)$ & & & \\
  $\ex(n,\crossing,\ears)$ & & & \\
  $\ex(n,\crossing,\swords)$ & $\Omega(n)$ & $O(n\log n)$ & \thmref{crossing-swords} \\$\ex(n,\crossing,\david)$ & & & \\
  $\ex(n,\ears,\swords)$ & \multicolumn{2}{c}{$\Theta(n^2$)} & \thmref{swords-ears-david} \\
  $\ex(n,\ears,\david)$ & \multicolumn{2}{c}{$\Theta(n^2$)} & \thmref{swords-ears-david} \\
  $\ex(n,\swords,\david)$ & \multicolumn{2}{c}{$\Theta(n^2$)} & \thmref{swords-ears-david} \\
\\ \hline
\end{tabular}

\section{Points of View and Easy Results}

Since there are eight possible forbidden configurations, there are
$2^8=256$ sets, $X$ for which we can study $\ex(n,X)$.  In this section
we present a few easy results that help reduce this number and we describe
different variants of the problem, some of which are easier to work with.
We begin with a result that cuts the number of interesting sets $X$
in half.

\subsection{Edge-Sharing Non-Overlapping Triangles are Irrelevant}

The following lemma shows that including the $\taco$ confguration in
the set $X$ of forbidden configurations has no effect on the asymptotics
of $\ex(n,X)$.

\begin{lem}\lemlabel{xcup}
   For any $X$, $\ex(n,X\cup\{\mariposa\}) \ge \ex(n,X)/8$.
\end{lem}

\begin{proof}
  Let $S$ be a set of triangles that achieves $\ex(n,X)$. For each pair
  of vertices $u$ and $w$ independently and uniformly choose a direction
  $\overrightarrow{uw}$ or $\overleftarrow{uw}$.  We then obtain a set
  $S'\subseteq S$ by removing any triangle that has a directed edge for
  which the triangle is to the left of the edge.  Observe that the set
  $S'$ does not contain a $\mariposa$ configuration.

  For any particular triangle $t\in S$, the probability that $t\in S'$
  is exactly $1/8$ since each of $t$'s three edges much be directed
  clockwise and edge directions are chosen independently.  By linearity of
  expectation, $\E[|S'|]=|S|/8=\ex(n,X)/8$.  We conclude therefore that
  there exists some subset $S''\subseteq S$ of size least $\ex(n,X)/8$
  that does not contain a $\mariposa$ configuration.  The set $S''$ proves
  that $\ex(n,X\cup\{\mariposa\}) \ge \ex(n,X)/8$.
\end{proof}

\subsection{The Top/Bottom View}

It will be helpful to consider an top/bottom variant of $\ex(n,X)$ that
is defined as follows (see \figref{top-bottom}).  Partition the vertices
of a convex $n$-gon using a horizontal line into a \emph{top half} of size
$\lceil n/2\rceil$ and a \emph{bottom half} of size $\lfloor n/2\rfloor$.
We define $\ex'(n,X)$ analogously to $\ex(n,X)$ except that we only count
triangles having one vertex in the bottom half and two vertices in the
top half.  When studying $\ex'$, each triangle we count has a naturally
defined \emph{bottom vertex} in the bottom half and a \emph{left vertex}
and \emph{right vertex}, each in the top half.

\begin{figure}
  \begin{center}
    \includegraphics{figs/left-right}
  \end{center}
  \caption{$\ex'$ only counts triangles with two vertices in the top half
     and one vertex in the bottom half.}
  \figlabel{top-bottom}
\end{figure}

The following lemma shows that we can, without losing much, we can study
$\ex'(n,X)$ instead of $\ex(n,X)$.

\begin{lem}\lemlabel{top-bottom}
  If $\ex'(n,X)\in O(n^c)$, then
  \[
     \ex(n,X)\in 
        \begin{cases} 
            O(n^c)     & \text{if $c>1$} \\
            O(n\log n) & \text{if $c=1$}
        \end{cases}
  \]
\end{lem}

\begin{proof}
   Let $S$ be a set of triangles that avoids $X$.  Every triangle in $S$
   is of one of the following types:
   \begin{enumerate}
      \item It has one vertex in the top half and two in the bottom half;
        there are $O(n^{c})$ such triangles.
      \item It has two vertices in the top half and one in the bottom
        half; there are $O(n^{c})$ such triangles.
      \item It has all three vertices in the top half; there are at most
        $\ex(\lceil n/2\rceil,X)$ such triangles.
      \item It has all three vertices in the bottom half; there are at
        most $\ex(\lfloor n/2\rfloor,X)$ such triangles.
   \end{enumerate}
   Thus, we obtain the recurrence inequality:
   \[  \ex(n,X) = O(n^{c}) + \ex(\lceil n/2\rceil,X) + \ex(\lfloor n/2\rfloor,X) \]
   which resolves to $O(n^c)$ for $c>1$ and $O(n\log n)$ for $c=1$.
\end{proof}


\subsection{The Dot-Puzzle View}

The top-bottom version of the problem gives us a sense of orientation,
but is still difficult to visualize the sets of triangles obtained this
way. Next, we show that there is a corresponding puzzle that is easy
to visualize.

Refer to \figref{point-view}.  In this puzzle, we are given $\binom{n}{2}$
points,
\[
    Q = \{(x,y): y\in\{1,\ldots,n-1\}, x\in\{y+1,\ldots,n-1\} \} \enspace .
\]
These points model the top/bottom view on a set of size $2n$, where the
point $(x,y)$ represents a triangle whose vertices are some point on
the bottom and the $x$th and $y$th points on the top, where the top
vertices are labelled $1,\ldots,n$ from left to right.

\begin{figure}
   \begin{center}
      \includegraphics{figs/point-view}
   \end{center}
   \caption{The Dot-Puzzle View of the Top/Bottom View. In this example,
     four rounds of the Dot-Puzzle have been played.}
   \figlabel{point-view}
\end{figure}

The dot-puzzle proceeds in $n$ rounds and during the $i$th round, the
player selects a set $Q_i\subseteq Q$ subject to certain constraints
that depend on the points selected in rounds $1,\ldots,i-1$.  In the
top/bottom view, the $i$th round determines which pairs of top vertices
form a triangle with the $i$th bottom vertex, where the bottom
vertices are labelled $1,\ldots,n$ from right to left.  

Of course, the constraints on which points can be selected during round
$i$ depend on the set of forbidden configurations and $\bigcup_{j=1}^{i-1}
Q_i$.  By proving bounds on $\sum_{i=1}^n |Q_i|$ we obtain bounds on
the maximum number of triangles obtained in the top-bottom view.

\Figref{forbidden-color}.a shows restrictions on the locations of points
placed during a single round.  It is interpreted as follows:  If the
central point, $p=(x,y)$, is placed during round $i$, and we wish to
avoid some particular configuration, $c$, then we should not place any
points in the parts of the figure that are have label $c$.  For instance,
if we wish to avoid the configuration $c=\taco$ configuration, then we
should not place any points in the same row or column as $p$; such a point
creates a $\taco$ configuration in which the shared edge
joins a bottom vertex to a left (same row) or right (same column) vertex.

\Figref{forbidden-color}.b shows the constraints placed on the locations
of points placed in subsequent rounds.  Its interpretation is similar
\figref{forbidden-color}.a. For example, if we wish to avoid a
$\nested$ configuration and we place the central point, $p$, during round
$i$, then, in every round $j>i$, we should not place any point directly
to the left or directly below $p$.  Any such point creates 
a $\nested$ configuration in which the shared vertex is the left vertex
(to the left of $p$) or the right vertex (below $p$) of both triangles.

%One caveat worth noting is that, unless $\taco$ is a forbidden
%configuration, the same point can be chosen in different rounds.

%\begin{table}
%\begin{center}
%\begin{tabular}{m{1ex}|>{\centering\arraybackslash}m{.45\textwidth}|>{\centering\arraybackslash}m{.45\textwidth}}
%      & killed by $(x,y)$ in later rounds 
%         & killed by $(x,y)$ in current round \\ \hline
%$\mariposa$ & \includegraphics[scale=.8]{figs/killers-1} \break% 
%           $\{\}$  
%         & \includegraphics[scale=.8]{figs/killersb-1} \break% 
%           $\{(x',x)\} \cup \{(y,y')\}$ \\ 
%$\taco$ & \includegraphics[scale=.8]{figs/killers-2} \break% 
%           $\{(x,y)\}$ 
%         & \includegraphics[scale=.8]{figs/killersb-2} \break% 
%           $\{(x,y')\} \cup \{(x',y)\}$ \\
%$\bat$ & \includegraphics[scale=.8]{figs/killers-3} \break% 
%             $\{(x',y) : x'<x \}\cup\{(x,y') : y'<y \}$ 
%           & \includegraphics[scale=.8]{figs/killersb-3} \break%
%             $\{(x',y'): x' < y\text{ or } y'> x\}$ \\
%$\nested$ &  \includegraphics[scale=.8]{figs/killers-4} \break%
%              $\{(x',y) : x'<x \}\cup\{(x,y') : y'<y \}$ 
%         &  \includegraphics[scale=.8]{figs/killersb-4} \break%
%            $\{(x',y'):\sign(x'-x)=\sign(y-y')\}$ \\
%$\crossing$ &  \includegraphics[scale=.8]{figs/killers-5} \break% 
%              $\{(x',y) : x'>x \}\cup\{(x,y') : y'>y \}$ 
%         &  \includegraphics[scale=.8]{figs/killersb-5} \break% 
%            $\{(x',y'):\sign(x'-x)=\sign(y'-y)\}$ \\
%$\ears$ &  \includegraphics[scale=.8]{figs/killers-6} \break%
%                $\{(x',y'): x'< y\}$
%         & \includegraphics[scale=.8]{figs/killersb-6} \break%
%           $\{\}$ \\
%$\swords$ & \includegraphics[scale=.8]{figs/killers-7} \break%
%               $\{(x',y'): x'>x, y'< y\}$
%         & \includegraphics[scale=.8]{figs/killersb-7} \break%
%           $\{\}$ \\
%$\david$ &\includegraphics[scale=.8]{figs/killers-8} \break%
%               $\{(x',y'): y < y' <x,\,\, x'>x\}$
%         & \includegraphics[scale=.8]{figs/killersb-8} \break%
%           $\{\}$ \\
%\end{tabular}
%\end{center}
%   \caption{The restrictions placed on the dot puzzle when for each of 
%     the forbidden subconfigurations.}
%   \tablabel{forbidden}
%\end{table}
%

\begin{figure}
   \begin{center}
      \newlength{\ka}
      \setlength{\ka}{\textwidth}
      \addtolength{\ka}{-1cm}
      \begin{tabular}{c@{\hspace{1cm}}c}
        \includegraphics[width=.48\ka]{figs/crapper-2} & 
        \includegraphics[width=.48\ka]{figs/crapper-1} \\
        (a) & (b)
      \end{tabular}
   \end{center}
   \caption{The regions killed by forbidden configurations during (a)~the current round and (b)~subsequent rounds.}
   \figlabel{forbidden-color}
\end{figure}


\subsection{Some Easy Bounds}

We say that a point set is \emph{non-decreasing} (respectively,
increasing, non-increasing, decreasing) if, when sorted lexicographically,
the $y$ coordinates of the points form a non-decreasing (respectively,
increasing, non-increasing, decreasing) sequence.

From \figref{forbidden-color}, some previous upper bounds naturally
fall out.  For example, Bra\ss's results \cite{brass:turan} that
$\ex(\nested)\in O(n^2)$ comes from the fact points selected during a
single round of the dot puzzle must be non-decreasing, and thus at most
$2n-3$ points can be selected take part in $Q_i$ during a round $i$.
Thus $\sum_{j=1}^{n}|Q_i| \le 2n^2-3n$, and the bound $\ex(\nested)\in
O(n^2)$ immediately follow from \lemref{top-bottom}.  Bra\ss's result
that $\ex(n,\crossing)\in O(n^2)$ is obtained in a similar way, using
the fact that the points in any $Q_i$ must be non-increasing.

Similarly, we can almost recover the result of Bra\ss, Rote and
Swanepoel \cite{brass.rote.ea:triangles} on $\ex(\ears, \swords,
\bat,\nested)$. 
By forbidding $\nested$, we have the rule
\begin{center}
  \includegraphics{figs/killersb-4} \enspace ,
\end{center}
which ensures that the set of points taken during a single round form
a non-decreasing point set.  From the forbidden configurations $\bat$,
$\nested$, $\ears$, and $\swords$, we obtain the rule
\begin{center}
  \includegraphics{figs/killers-9}
\end{center}
for which points are disallowed in subsequent rounds.  This rule ensures
that, after round $i$ any points chosen are either the topmost-rightmost
point in $Q_i$ or are above and to the right of this point.  Taken
together, these rules imply that
\[
    \sum_{i=1}^n|Q_i| \le 3n-2 \enspace ,
\]
Since the union of $Q_i$ is a non-decreasing point set (whose size is
therefore at most $2n-2$), and each $Q_i$ shares at most one point with
$Q_{i+1}$.  The bound $\ex(\ears, \swords, \bat,\nested)\in
O(n\log n)$ then follows from \lemref{top-bottom}.

\section{New Results}

% The following is the start of some bullshit argument, unless I can fix it.

%Earlier, we showed that the inclusion of the $\mariposa$ configuration in
%the set of excluded configurations, $X$, has no effect on the asymptotics
%of $\ex(n,X)$.  Here we show that a similar, though slightly weaker,
%result for the $\nested$ configuration.
%
%\begin{lem}
%  $\ex'(n,\{\taco,\bat\}\cup X) \in \Omega(\ex'(n,\{\taco\}\cup X)/\log n)$.
%\end{lem} 
%
%\begin{proof}[Proof Sketch]
%    Assume without loss of generality that $n$ is a power of 2 Partition
%    $Q$ into $O(\log n)$ \emph{layers}, where layer 0 contains
%    the \emph{square}, $L_0=\{(x,y)\in Q: x\ge n/2,\, y\le n/2\}$.
%    Subsequent layers are obtained by recursing on the two triangles in
%    $Q\setminus L_0$, so that $L_1$ contains two squares, $L_2$ contains
%    four squares, and so on.
%
%    Let $Q_1,\ldots,Q_n$ be a play of the dot-puzzle that achieves
%    $\ex'(n,\{\taco\}\cup X)$ while avoiding $\{\taco\}\cup X$
%    configurations.  Observe that, because we exclude $\taco$, the sets
%    $Q_i$ and $Q_j$ are disjoint for all $1\le i<j\le n$.
%
%    Then, for any layer $L_i$, the sets $Q_1\cap L_i,\ldots,Q_n\cap L_i$
%    avoid all configurations in $\{\taco\}\cup X$ as well as $\bat$
%    configurations.  By the pigeonhole principle, one of these layers,
%    say $L_i$, has size $\Omega(\ex'(n,\{\taco\}\cup X)/\log n)$.
%
%
%
%points that maximize
%\end{proof}
%
%is irrelevan
%

After this warm-up, and with the dot-puzzle view, we are ready to prove
some new results.  We begin with a collection of results on forbidding
the $\swords$ configuration.

\subsection{Forbidding Crossing Vertex-Disjoint Triangles}

In the following, we use the notation $\ymin(S)$ denote the minimum
$x$-coordinate of any point in the point set $S$.  If $S$ is empty,
then we define $\xmin(S)$ as $n+1$.  We define $\ymin(S)$, $\xmax(S)$,
and $\ymax(S)$ similarly, except that $\ymin(\emptyset) = n$,
$\ymax(\emptyset)=0$, and $\xmax(\emptyset)=1$.


\begin{obs}\obslabel{swords-region}
  For any non-empty set $S\subset Q$, $\survivors(\{\swords\},S)$
  is contained in the union of a \emph{triangular region} $T=\{(x,y)\in
  Q: x\le \ymin(S)+1\}$, a \emph{rectangular region} $R=\{(x,y)\in Q:
  x\ge \xmax(S),\, y\ge \ymax(S)\}$, a \emph{vertical line} $V=\{(x,y)\in Q:
  x=\xmin(S)\}$ and a \emph{horizontal line} $H=\{(x,y)\in Q: y=\ymin(S)\}$.
\end{obs}

\obsref{swords-region} gives us a form of \emph{potential function}
that we can use.  We will apply it with $S=\bigcup_{j=1}^i Q_i$ so that as
rounds proceed, any time the value of $\xmax(S)$ or $\ymax(S)$ increase,
the size of the rectangle $R$ decreases. Any time the value of $\ymin(S)$
decreases, the size of the triangle $T$ decreases.  So that we can refer
to these point sets as they change, we let $S_i=\bigcup_{j=1}^i Q_i$ and
we define $T_i$, $R_i$, $V_i$, and $H_i$ as in \obsref{swords-region},
but with respect to the set $S=S_i$.

\begin{thm}\thmlabel{taco-swords}
  $\ex(n,\{\taco,\swords\}) \in O(n\log n)$.
\end{thm}

\begin{proof}
  We will directly bound $\sum_{i=1}^n |Q_i|$.  We begin by noting that,
  since we forbid $\taco$, the set of points in each $Q_i$ contains
  at most one point from each row and each column.  We will use this
  fact implicitly for the rest of the proof.

  Now, observe that if $Q_i$ contains $t_i\ge 2$ points in $T_i$, then
  $\ymin(S_{i+1}) \le \xmin(S_i)-t_i+2\}$.  This immediately
  implies that $\sum_{i=1}^n t_i \le 3n$.

  Similarly, if $Q_i$ contains $r_i\ge 1$ points in $R_i$, then
  $\ymax(S_{i+1})\ge \ymax(S_i)+r_i-1$.  This immediately
  implies that $\sum_{i=1}^n r_i\le 2n$.
  
  Finally, we observe that $Q_i$ contains at most one point each from
  $H_i$ and $V_i$, so
  \[
      \sum_{i=1}^{n}|Q_i| \le \sum_{i=1}^n(t_i+r_i+2) \le 7n \enspace .
  \]
  We have just shown that $\ex'(n,,\{\swords,\taco\})\in O(n)$,
  so the theorem follows from \lemref{top-bottom}.
\end{proof}

\begin{thm}\thmlabel{swords-bat}\thmlabel{bat-swords}
  $\ex(n,\{\swords,\bat\}) \in \Theta(n^2)$.
\end{thm}

\begin{proof}
   The upper bound follows from the $O(n^2)$ upper bound on
   $\ex(n,\{\swords\})$ \cite{brass:turan}.
   For the lower-bound, we set $Q_1=\{(x,y)\in Q: x>n/2, y<n/2\}$ and
   set $Q_2=Q_3=\cdots Q_n=\emptyset$.
\end{proof}

\begin{thm}\thmlabel{nested-swords}
  $\ex'(n,\{\nested,\swords\}) \in O(n)$.
\end{thm}

\begin{proof}
   The proof is similar to the proof of \thmref{swords-edgea}.
   The forbidden configuration $\nested$ implies that each set $Q_i$
   must be non-decreasing (see \figref{forbidden-color}a).
   We will use this fact implicitly for the rest of the proof.

   Now, if $Q_i$ contains $t_i$ points in $T_i$, then the fact
   that $Q_i$ is non-decreasing implies that $\ymin(S_{i+1})\le
   \ymin(S_i)-\lfloor t_i/2 \rfloor$.  This immediately implies that
   $\sum_{j=1}^n t_i \le 3n$.

   Similarly, if $Q_i$ contains $r_i\ge 1$ points in $R_i$, then the fact
   that $Q_i$ is non-decreasing means that
   \[
       \ymax(S_{i+1})+\xmax(S_{i+1}) 
             \ge \ymax(S_{i})+\xmax(S_{i}) + r_i - 1 \enspace .
   \]
   This immediately implies that $\sum_{j=1}^n r_i \le 3n$.

   To account for points of $Q_i$ on the vertical line $V_i$, we consider
   the lowest point of $\survivors(\{\swords\},Q_i)$ on  
   $V_i\setminus T_i\setminus R_i$ and denote the $y$-coordinate of this
   point by $y_i$.  If $V_i\setminus T_i\setminus R_i$ contains $v_i\ge
   1$ points of $Q_i$, then $y_{i+1}\ge y_i+v_i-1$.  This immediately
   implies that $\sum_{j=1}^n v_i \le 2n$.

   Accounting for points of $Q_i$ on the horizontal line $H_i$ is similar,
   but with respect to the left-most point of 
   $\survivors(\{\swords\},Q_i)$ in $H_i\setminus T_i\setminus R_i$.

   Summing everything, we find that
   $\ex'(n,\{\swords,\nested\})\le\sum_{j=1}^n |Q_i|\le 10n$, and
   we finish by applying \lemref{top-bottom}.
\end{proof}

\begin{thm}\thmlabel{crossing-swords}
  $\ex'(n,\{\crossing,\swords\}) \in O(n)$.
\end{thm}

\begin{proof}
  TODO. Similar to previous proof.
\end{proof}

\begin{thm}\thmlabel{swords-ears-david}
  $\ex(n,\{\swords,\ears,\david\}) \in \Theta(n^2)$.
\end{thm}

\subsection{Disorganized Results}

Now, with our tools in hand, we are ready to study some problems.

\begin{thm}\thmlabel{blech}
  $\ex(n,\{\taco,\nested,\crossing\}) \in O(n\log n)$.
\end{thm}

\begin{proof}
  Since we forbid $\nested$, we know that the set of points taken
  during each round is non-decreasing.  Taking the union of the rules
  for $\taco$, $\nested$, and $\crossing$, we obtain the rule
  \begin{center}
     \includegraphics{figs/killers-10}
  \end{center}
  which ensures that during subsequent rounds we can not take a point
  from any column or row used in the previous round.

  Imagine scanning the points in $Q_i$ in non-decreasing order.  Combining
  the two preceding observations, we see that each point we scan is in
  a row and column not used before in $Q_1,\ldots,Q_{i-1}$ and it is
  on a row or a column that has not appeared before in the scan order.
  Thus, each point of $Q_i$ that we scan kills one row and/or one column
  so that this row/column can not be used in $Q_{i+1},\ldots,Q_n$.  Therefore,
  $\sum_{j=1}^n |Q_i| \le 2n-3$.
\end{proof}

\begin{thm}
  $\ex'(n,\{\nested,\david\})\in \Omega(n^2)$.
\end{thm}

\begin{proof}[Proof Sketch]
  Repeatedly take points on the diagonal.
\end{proof}

\begin{conj}
  $\ex'(n,\{\taco,\nested,\david\})\in O(n)$.
\end{conj}



\bibliographystyle{plain}
\bibliography{turan}

\end{document}


